Science advances in no small part part through the combinatorial
interaction of ideas. As in the economy at large, specialization and
comparative advantage play an increasing role in today's hyper-complex
scientific domains. The key to progress is often the right pair of eyes, on
the right information, at the right time, in the context of the right
problem. Biomedical breakthroughs, for example, have frequently come from
obscure corners of the research establishment. CRISPR interference, now one
of the methods being evaluated for efficient genome editing, were found by
researchers at Danisco~\cite{barrangou2007crispr}, a yogurt company.

@article{barrangou2007crispr,
  title={CRISPR provides acquired resistance against viruses in
prokaryotes},
  author={Barrangou, Rodolphe and Fremaux, Christophe and Deveau,
H{\'e}l{\`e}ne and Richards, Melissa and Boyaval, Patrick and Moineau,
Sylvain and Romero, Dennis A and Horvath, Philippe},
  journal={Science},
  volume={315},
  number={5819},
  pages={1709--1712},
  year={2007},
  publisher={American Association for the Advancement of Science}
}

There is considerable anecdotal evidence to suggest that scientific
breakthroughs have historically been bottlenecked in part by the slow rate
of convergence -- i.e., spreading and re-integration in new contexts -- of
knowledge embodied in the literature.

\begin{itemize}

\item Khorana described the core procedure of the polymerase chain reaction
(PCR) in 1971~\cite{kjeppe1971studies}, 14 years before its re-discovery in
1985-1988, which led to a Nobel prize~\cite{saiki1985enzymatic,
mullis1987specific, saiki1988primer}:\emph{``The DNA duplex would be
denatured to form single strands. This denaturation step would be carried
out in the presence of a  sufficiently large excess of the two appropriate
primers. Upon cooling, one would hope  to obtain two structures, each
containing the full length of the template strand  appropriately complexed
with the primer. DNA polymerase will be added to complete the process of
repair replication. Two molecules of the original duplex should result. The
whole cycle could be repeated, there being added every time a fresh dose of
the  enzyme. It is however, possible that upon cooling after denaturation
of the DNA  duplex, renaturation to form the original duplex would
predominate over the template-  primer complex formation. If this tendency
could not be circumvented by adjusting  the concentrations of the primers,
clearly one would have to resort to the separation  of the strands and then
carry out repair replication. After every cycle of repair  replication, the
process of strand separation would have to be repeated. Experiments based
on these lines of thought are in progress.''}~\cite{kjeppe1971studies}, a
study cited only 312 times.

\item Rhodopsin-based optogenetics was demonstrated in yeast mitochondria
in 1994~\cite{hoffmann1994photoactive}, in a paper cited only 8 times, 11
years before its re-discovery and application to neuroscience in
2005~\cite{boyden2005millisecond}, a step described in the citation for the
Brain Prize as ``the most important technical advance in neuroscience in
the past 40 years''.

\item Green fluorescent protein (GFP) was isolated from
jellyfish~\cite{shimomura1962extraction,morise1974intermolecular} and
studied in the 1960s and 1970s, but only expressed in other organisms by
two independent groups in
1992-1994~\cite{chalfie1994green,inouye1994aequorea}, leading to a Nobel
prize and a ubiquitous, essential tool in all areas of life science.

\item Other examples include the study of RNA interferences (RNAi)
mechanisms in petunias in the early 1990s~\cite{jorgensen1990altered}, long
before their use in other organisms in 1998~\cite{fire1998potent} (another
Nobel prize), despite their being no major technical obstacles to doing so.

\end{itemize}

\emph{What if we could make the combination of obscure but important ideas,
specialized knowledge and unique skills occur more quickly?}

@article{kjeppe1971studies,
  title={Studies on polynucleotides. Repair replication of short synthetic
DNAs as catalyzed by DNA polymerase},
  author={KJeppe, K and Ohtsuka, E and Kleppe, R and Molineux, I and
Khorana, HG},
  journal={J. Mol. Biol},
  volume={56},
  pages={341--361},
  year={1971}
}

@article{saiki1985enzymatic,
  title={Enzymatic amplification of beta-globin genomic sequences and
restriction site analysis for diagnosis of sickle cell anemia},
  author={Saiki, Randall K and Scharf, Stephen and Faloona, Fred and
Mullis, Kary B and Horn, Glenn T and Erlich, Henry A and Arnheim, Norman},
  journal={Science},
  volume={230},
  number={4732},
  pages={1350--1354},
  year={1985},
  publisher={American Association for the Advancement of Science}
}

@article{mullis1987specific,
  title={Specific synthesis of DNA in vitro via a polymerase-catalyzed
chain reaction.},
  author={Mullis, Kary B and Faloona, Fred A},
  journal={Methods in enzymology},
  volume={155},
  pages={335},
  year={1987}
}

@article{saiki1988primer,
  title={Primer-directed enzymatic amplification of DNA with a thermostable
DNA polymerase},
  author={Saiki, Randall K and Gelfand, David H and Stoffel, Susanne and
Scharf, Stephen J and Higuchi, Russell and Horn, Glenn T and Mullis, Kary B
and Erlich, Henry A},
  journal={Science},
  volume={239},
  number={4839},
  pages={487--491},
  year={1988},
  publisher={American Association for the Advancement of Science}
}

@article{hoffmann1994photoactive,
  title={Photoactive mitochondria: in vivo transfer of a light-driven
proton pump into the inner mitochondrial membrane of Schizosaccharomyces
pombe},
  author={Hoffmann, Astrid and Hildebrandt, Volker and Heberle, Joachim and
B{\"u}ldt, Georg},
  journal={Proceedings of the National Academy of Sciences},
  volume={91},
  number={20},
  pages={9367--9371},
  year={1994},
  publisher={National Acad Sciences}
}

@article{boyden2005millisecond,
  title={Millisecond-timescale, genetically targeted optical control of
neural activity},
  author={Boyden, Edward S and Zhang, Feng and Bamberg, Ernst and Nagel,
Georg and Deisseroth, Karl},
  journal={Nature neuroscience},
  volume={8},
  number={9},
  pages={1263--1268},
  year={2005},
  publisher={Nature Publishing Group}
}

@article{shimomura1962extraction,
  title={Extraction, purification and properties of aequorin, a
bioluminescent protein from the luminous hydromedusan, Aequorea},
  author={Shimomura, Osamu and Johnson, Frank H and Saiga, Yo},
  journal={Journal of cellular and comparative physiology},
  volume={59},
  number={3},
  pages={223--239},
  year={1962},
  publisher={Wiley Online Library}
}

@article{morise1974intermolecular,
  title={Intermolecular energy transfer in the bioluminescent system of
Aequorea},
  author={Morise, Hiroshi and Shimomura, Osamu and Johnson, Frank H and
Winant, John},
  journal={Biochemistry},
  volume={13},
  number={12},
  pages={2656--2662},
  year={1974},
  publisher={ACS Publications}
}

@article{chalfie1994green,
  title={Green fluorescent protein as a marker for gene expression},
  author={Chalfie, Martin and Tu, Yuan and Euskirchen, Ghia and Ward,
William W and Prasher, Douglas C},
  journal={Science},
  volume={263},
  number={5148},
  pages={802--805},
  year={1994},
  publisher={American Association for the Advancement of Science}
}

@article{inouye1994aequorea,
  title={Aequorea green fluorescent protein: Expression of the gene and
fluorescence characteristics of the recombinant protein},
  author={Inouye, Satoshi and Tsuji, Frederick I},
  journal={FEBS letters},
  volume={341},
  number={2},
  pages={277--280},
  year={1994},
  publisher={Elsevier}
}

@article{jorgensen1990altered,
  title={Altered gene expression in plants due to trans interactions
between homologous genes},
  author={Jorgensen, Richard},
  journal={Trends in biotechnology},
  volume={8},
  pages={340--344},
  year={1990},
  publisher={Elsevier}
}

@article{fire1998potent,
  title={Potent and specific genetic interference by double-stranded RNA in
Caenorhabditis elegans},
  author={Fire, Andrew and Xu, SiQun and Montgomery, Mary K and Kostas,
Steven A and Driver, Samuel E and Mello, Craig C},
  journal={nature},
  volume={391},
  number={6669},
  pages={806--811},
  year={1998},
  publisher={Nature Publishing Group}
}

And some more quotes:

"...the scientist may know a little patch of something... he may know a few
spots from other people's work... he may even be able to read a book...

...almost everything that's known to man, he doesn't know anything about...
and that's because it's gotten a bit complicated...

...occasionally a man knows two things, and that intersection may be a
great event in the history of ideas...

...occasionally, a man may think that something is relevant or exciting
which no one before thought concerned him professionally, and that may
change the history of the world..." --J. Robert Oppenheimer

\begin{fquote}[Jacques Monod]Ideas have retained some of the properties of
organisms. Like them, they tend to perpetuate their structure and to breed;
they too can fuse, recombine, segregate their content.\end{fquote}

\begin{fquote}[Kevin Kelly][Speculation on the Next 100 Years of
Science][2006~\cite{kelly2006speculations}]Science will continue to
surprise us with what it discovers and creates; then it will astound us by
devising new methods to surprise us.\end{fquote}

"There is a growing mountain of research. But there is increased evidence that we are being bogged down today as specialization extends. The investigator is staggered by the findings and conclusions of thousands of other workers�conclusions which he cannot find time to grasp, much less to remember, as they appear. Yet specialization becomes increasingly necessary for progress, and the effort to bridge between disciplines is, correspondingly, superficial.

Mendel's concept of the laws of genetics was lost to the world for a generation because his publication did not reach the few who were capable of grasping and extending it; and this sort of catastrophe is undoubtedly being repeated all about us, as truly significant attainments become lost in the mass of the inconsequential.

-- Vannevar Bush 1945, As We May Think"